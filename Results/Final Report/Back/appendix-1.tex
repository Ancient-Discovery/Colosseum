\chapter{Currently Unused Data}
\label{ch:appendix_1}
There are some data that we collected but did not use for some reasons. Anyone who is interested can find them in our project GitHub repository (in \texttt{Data/Unused} directory); meanwhile, there are some useful websites that we found during this tise without using frequently and we hope that the public sharing can reduce time for gathering information if someone will conduct similar researches like what we did. Therefore we list them below:
\begin{itemize}
	\item More than 60000 processed seal script character images from Liu Shu Tong and Shuowen Jiezi are packed along with oracle bone script and bronze script data that we used in .zip format; ancient Chinese font files in .ttf format from various sources. Details can be found in \texttt{README.txt} file.
	\item Two formal proposals from Taipei Computer Association	(TCA) and China on tentative allocation of Plane 3, the TIP (Tertiary Ideographic Plane) including distinct 10982 small seal scripts (divided into 14 volumes) and 1463 oracle bone scripts to be standardised.
	\item \url{http://humanum.arts.cuhk.edu.hk/Lexis/lexi-mf/} is called Multi-function Chinese Character Database, maintained by The Chinese University of Hong Kong. It provides a structured display for each character; also a special function of English-Chinese lookup is offered. It is language-friendly for Mandarin, English and even Cantonese dialect users.
	\item \url{http://xiaoxue.iis.sinica.edu.tw/} (in traditional Chinese) is a Chinese character database built by Institute of Information Science, Academia Sinica; it claims that it collects more than 200 thousand glyphs from diverse sources.
\end{itemize}