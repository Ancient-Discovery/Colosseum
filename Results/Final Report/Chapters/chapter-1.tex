\chapter{Introduction}\label{ch:introduction}
Oracle bone script is one of the earliest known Chinese writing systems, dating back to around 3000 years ago, which were engraved on animal scapulae or turtle plastrons (flat under-part of shells) with sharp stones or bronze instruments. Since the first piece of oracle bone scripts was found and catalogued by scholars in Anyang, the primary capital of the late Shang Dynasty, which ensured undoubtedly the authenticity of existence of pre-Zhou-Dynasty history, hundreds of thousands of bone fragments were dug along with numerous artefacts on site in subsequent archaeological excavations. They contain more than 30000 distinct symbols, estimated to be variant forms of around 4600 identified individual characters; these new materials are of importance for Chinese palaeography and research into prehistoric civilisation. Majority of oracle bones at that time was associated with pyromancy (divination by fire), by inscribing questions on shoulder blades, burning them and interpreting occurring cracks formed on the surface as divine responses, which were engraved on the polished surface alongside the cracks. And analysis indicates that most of text constitutes hunting expeditions, weather, warfare, astronomical events, administrative orders and selection of auspicious for important ceremonies\cite[pp.~5]{de2000sources}. Oracle bone scripts are almost identical with modern functional writing system in use currently; highly advanced and surprisingly mature, oracle bone scripts achieve a very high extent of abstraction in terms of expressing meanings. A whole scientific discipline called Oracle Bone Studies was set up, dedicating to deciphering them and reconstructing lives of Shang residents or confirming succession of kings. Compared to bronze scripts later on, preceding oracle bone scripts appear to be of raw simplicity, obviously because the difficulty of engraving symbols on materials with firm surface could not be overcome easily at that time.

However, just 1300 ones can be translated to their modern counterparts by linguists. Occasionally many recognised characters are still disputed due to lack of definitive explanations. Structures and orientations of each character varied largely in different time periods, which sets an even higher barrier for researchers. Deciphering one new symbol could reveal plenty of new facts in the ancient society.

And machine learning, a main sub-field of artificial intelligence, revived increasingly in recent years after nearly twenty years of AI winter due to tremendous hardware performance increase; it involves statistical techniques to improve performance of specific problems by feeding data instead of being explicitly programmed and its applications covers all industrial domains. After AlphaGo, a computer program developed by Google Deepmind, beat 9-dan player Lee Sedol without handicaps in March 2016, deep learning, one of classical machine learning algorithms inspired by biological nervous systems, gradually became well-known by public. The most major difference between deep learning and traditional machine learning is the scale of data. The more amount of data a program is fed in, the better results deep learning will achieve typically. As opposed to task-specific algorithms, models generated by deep learning can be transferred to various types of problems without completely rewriting code. And with assistance of deep learning, character recognition, a field of research in pattern matching, computer vision, and certainly artificial intelligence itself, has developed rapidly. And recognition based on image processing (correlation) and feature detection easily makes a breakthrough and avoids bottleneck even with high-end hardware.

In this project, we merged two separate disciplines together, via applying several latest deep learning techniques to speed up the process of recognising oracle bone script characters, which can be an potential academic helper for Chinese palaeography. And then we applied supervised deep learning, with both paired and unpaired data, to create two types of GANs, generative adversarial networks, in order to generate bronze inscription images from oracle bone script patterns or vice versa.